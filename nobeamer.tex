% Include eplain.tex if it is not yet included
\def\ifundefined#1{\expandafter\ifx\csname#1\endcsname\relax}
\def\ifdefined#1{\ifundefined{#1}\relax\else}
{\ifdefined{undefined}\errmessage{\noexpand\undefined cannot be defined!}\fi}
\ifx\eplain\undefined %
    \message{nobeamer: eplain not detected, loading eplain.} %
    \input eplain %
\fi 

\message{nobeamer: loading utf8plainmac.}
% Include utf8plainmac.tex, by egreg on StackOverflow: 163852.
% It allows input of UTF-8 characters in Plain TeX.
% Retrieved on 15-Dec-2021, modified.
%%%%%%%%%%%%%%%%%%%%%%%%%%%%%%%%%% START UTF8PLAINMAC %%%%%%%%%%%%%%%%%%%%%%%%%%%%%%%%%%%
% We set a safe catcode for ^ and ^^^; XeTeX uses the ^^^^ convention for
% specifying arbitrary 16 bit code points. So if XeTeX is used, \gobble
% eats up ^^^^0021, while with an 8 bit engine only ^^^ is
% swallowed and \next is not \empty. In the end, \ifunicode is \iftrue if
% the engine is Unicode aware, it is \iffalse if the engine is 8 bit.

\catcode`\^=7
\catcode`\~=\active 

\newif\ifunicodeengine
\begingroup
\catcode30=12 % just in case: 30 is `\^^^
\def\gobble#1#2{}
\edef\next{\gobble^^^^0021}
\expandafter\endgroup
\ifx\next\empty\unicodeenginetrue\else\unicodeenginefalse\fi

\message{Engine is \ifunicodeengine Unicode aware\else 8 bit\fi, loading UTF-8 combinations}

\ifunicodeengine
%%% Make the first argument active and define it as the fourth
%%% The trick avoids a global definition: the \lowercase changes
%%% ~ into #1 as active character; then \endgroup\def#1 is put
%%% back into the token stream (here #1 stands for the actual
%%% character given as argument); the same trick is used for
%%% \UseUnicodeCharacter, which must have an argument expressed
%%% as a four digit hexadecimal number (with uppercase A..F).
  \def\DoUTFCombination#1#2#3#4{\catcode"#1\active
    \begingroup\lccode`~="#1\lowercase{\endgroup\def~}{#4}}
  \def\UseUnicodeCharacter#1{\begingroup\lccode`~="#1\lowercase{\endgroup~}}
\else
%%% The UTF-8 prefixes are made active; they just look at the
%%% following token, which is a category 12 character unless something
%%% strange has happened, and forms with it a control sequence that
%%% will be defined later
  \catcode`\^^c2=\active
  \def^^c2#1{\csname UTFprefix-c2#1\endcsname}
  \catcode`\^^c3=\active
  \def^^c3#1{\csname UTFprefix-c3#1\endcsname}
  \catcode`\^^c4=\active
  \def^^c4#1{\csname UTFprefix-c4#1\endcsname}
  \catcode`\^^c5=\active
  \def^^c5#1{\csname UTFprefix-c5#1\endcsname}
  \catcode`\^^c6=\active
  \def^^c6#1{\csname UTFprefix-c6#1\endcsname}
  \catcode`\^^c7=\active
  \def^^c7#1{\csname UTFprefix-c7#1\endcsname}
  \catcode`\^^c8=\active
  \def^^c8#1{\csname UTFprefix-c8#1\endcsname}
  \catcode`\^^cb=\active
  \def^^cb#1{\csname UTFprefix-cb#1\endcsname}

%%% If the file is input by a UTF-8 unaware engine, we define the main
%%% command that associates the UTF-8 character (actually a two byte
%%% combination) to a list of tokens; we define also
%%% \UseUnicodeCharacter to access the same replacement text via an
%%% auxiliary macro \UTFCodePoint-xxxx, where xxxx stands for the
%%% argument to \UseUnicodeCharacter, a four digit hexadecimal number
%%% (uppercase A..F).
  \def\DoUTFCombination#1#2#3#4{%
    \expandafter\def\csname UTFprefix-#2#3\endcsname{#4}%
    \expandafter\def\csname UTFCodePoint-#1\endcsname{#4}%
  }
  \def\UseUnicodeCharacter#1{\csname UTFCodePoint-#1\endcsname}
\fi

%%% Some (actually many) UTF-8 characters cannot be printed with T1
%%% or TS1 encoded fonts
\newif\ifUTFwarning \UTFwarningtrue
\def\BadUTF#1{%
  \ifUTFwarning
    \global\UTFwarningfalse
    \errhelp{Look in the log file for unsupported characters}%
    \errmessage{Unsupported UTF character}%
  \fi
  \wlog{Character #1 not currently supported on line
    \the\inputlineno}%
}

%%% A shorthand for choosing the text companion font
\def\tcsym#1{{\tcfont\char#1}}

%%% The list of characters: Unicode code point, prefix and second
%%% byte, then the definition.
\DoUTFCombination{00A0}{c2}{^^a0}{~} % NO-BREAK SPACE
\DoUTFCombination{00A1}{c2}{^^a1}{!`} % INVERTED EXCLAMATION MARK
\DoUTFCombination{00A2}{c2}{^^a2}{\tcsym{"8B}} % CENT SIGN
\DoUTFCombination{00A3}{c2}{^^a3}{\pound} % POUND SIGN
\DoUTFCombination{00A4}{c2}{^^a4}{\tcsym{"A4}} % CURRENCY SIGN
\DoUTFCombination{00A5}{c2}{^^a5}{\tcsym{"A5}} % YEN SIGN
\DoUTFCombination{00A6}{c2}{^^a6}{\tcsym{"A6}} % BROKEN BAR
\DoUTFCombination{00A7}{c2}{^^a7}{\tcsym{"A7}} % SECTION SIGN
\DoUTFCombination{00A8}{c2}{^^a8}{\"{}} % DIAERESIS
\DoUTFCombination{00A9}{c2}{^^a9}{\tcsym{"A9}} % COPYRIGHT SIGN
\DoUTFCombination{00AA}{c2}{^^aa}{\tcsym{"AA}} % FEMININE ORDINAL INDICATOR
\DoUTFCombination{00AB}{c2}{^^ab}{>} % RIGHT-POINTING DOUBLE ANGLE QUOTATION MARK
\DoUTFCombination{00BC}{c2}{^^bc}{\tcsym{"BC}} % VULGAR FRACTION ONE QUARTER
\DoUTFCombination{00BD}{c2}{^^bd}{\tcsym{"BD}} % VULGAR FRACTION ONE HALF
\DoUTFCombination{00BE}{c2}{^^be}{\tcsym{"BE}} % VULGAR FRACTION THREE QUARTERS
\DoUTFCombination{00BF}{c2}{^^bf}{?`} % INVERTED QUESTION MARK

\DoUTFCombination{00C0}{c3}{^^80}{\`A} % LATIN CAPITAL LETTER A WITH GRAVE
\DoUTFCombination{00C1}{c3}{^^81}{\'A} % LATIN CAPITAL LETTER A WITH ACUTE
\DoUTFCombination{00C2}{c3}{^^82}{\^A} % LATIN CAPITAL LETTER A WITH CIRCUMFLEX
\DoUTFCombination{00C3}{c3}{^^83}{\~A} % LATIN CAPITAL LETTER A WITH TILDE
\DoUTFCombination{00C4}{c3}{^^84}{\"A} % LATIN CAPITAL LETTER A WITH DIAERESIS
\DoUTFCombination{00C5}{c3}{^^85}{\AA} % LATIN CAPITAL LETTER A WITH RING ABOVE
\DoUTFCombination{00C6}{c3}{^^86}{\AE} % LATIN CAPITAL LETTER AE
\DoUTFCombination{00C7}{c3}{^^87}{\c{C}} % LATIN CAPITAL LETTER C WITH CEDILLA
\DoUTFCombination{00C8}{c3}{^^88}{\`E} % LATIN CAPITAL LETTER E WITH GRAVE
\DoUTFCombination{00C9}{c3}{^^89}{\'E} % LATIN CAPITAL LETTER E WITH ACUTE
\DoUTFCombination{00CA}{c3}{^^8a}{\^E} % LATIN CAPITAL LETTER E WITH CIRCUMFLEX
\DoUTFCombination{00CB}{c3}{^^8b}{\"E} % LATIN CAPITAL LETTER E WITH DIAERESIS
\DoUTFCombination{00CC}{c3}{^^8c}{\`I} % LATIN CAPITAL LETTER I WITH GRAVE
\DoUTFCombination{00CD}{c3}{^^8d}{\'I} % LATIN CAPITAL LETTER I WITH ACUTE
\DoUTFCombination{00CE}{c3}{^^8e}{\^I} % LATIN CAPITAL LETTER I WITH CIRCUMFLEX
\DoUTFCombination{00CF}{c3}{^^8f}{\"I} % LATIN CAPITAL LETTER I WITH DIAERESIS
\DoUTFCombination{00D0}{c3}{^^90}{\DH} % LATIN CAPITAL LETTER ETH
\DoUTFCombination{00D1}{c3}{^^91}{\~N} % LATIN CAPITAL LETTER N WITH TILDE
\DoUTFCombination{00D2}{c3}{^^92}{\`O} % LATIN CAPITAL LETTER O WITH GRAVE
\DoUTFCombination{00D3}{c3}{^^93}{\'O} % LATIN CAPITAL LETTER O WITH ACUTE
\DoUTFCombination{00D4}{c3}{^^94}{\^O} % LATIN CAPITAL LETTER O WITH CIRCUMFLEX
\DoUTFCombination{00D5}{c3}{^^95}{\~O} % LATIN CAPITAL LETTER O WITH TILDE
\DoUTFCombination{00D6}{c3}{^^96}{\"O} % LATIN CAPITAL LETTER O WITH DIAERESIS
\DoUTFCombination{00D7}{c3}{^^97}{\tcsym{"D6}} % MULTIPLICATION SIGN
\DoUTFCombination{00D8}{c3}{^^98}{\O} % LATIN CAPITAL LETTER O WITH STROKE
\DoUTFCombination{00D9}{c3}{^^99}{\`U} % LATIN CAPITAL LETTER U WITH GRAVE
\DoUTFCombination{00DA}{c3}{^^9a}{\'U} % LATIN CAPITAL LETTER U WITH ACUTE
\DoUTFCombination{00DB}{c3}{^^9b}{\^U} % LATIN CAPITAL LETTER U WITH CIRCUMFLEX
\DoUTFCombination{00DC}{c3}{^^9c}{\"U} % LATIN CAPITAL LETTER U WITH DIAERESIS
\DoUTFCombination{00DD}{c3}{^^9d}{\'Y} % LATIN CAPITAL LETTER Y WITH ACUTE
\DoUTFCombination{00DE}{c3}{^^9e}{\TH} % LATIN CAPITAL LETTER THORN
\DoUTFCombination{00DF}{c3}{^^9f}{\ss} % LATIN SMALL LETTER SHARP S
\DoUTFCombination{00E0}{c3}{^^a0}{\`a} % LATIN SMALL LETTER A WITH GRAVE
\DoUTFCombination{00E1}{c3}{^^a1}{\'a} % LATIN SMALL LETTER A WITH ACUTE
\DoUTFCombination{00E2}{c3}{^^a2}{\^a} % LATIN SMALL LETTER A WITH CIRCUMFLEX
\DoUTFCombination{00E3}{c3}{^^a3}{\~a} % LATIN SMALL LETTER A WITH TILDE
\DoUTFCombination{00E4}{c3}{^^a4}{\"a} % LATIN SMALL LETTER A WITH DIAERESIS
\DoUTFCombination{00E5}{c3}{^^a5}{\aa} % LATIN SMALL LETTER A WITH RING ABOVE
\DoUTFCombination{00E6}{c3}{^^a6}{\ae} % LATIN SMALL LETTER AE
\DoUTFCombination{00E7}{c3}{^^a7}{\c{c}} % LATIN SMALL LETTER C WITH CEDILLA
\DoUTFCombination{00E8}{c3}{^^a8}{\`e} % LATIN SMALL LETTER E WITH GRAVE
\DoUTFCombination{00E9}{c3}{^^a9}{\'e} % LATIN SMALL LETTER E WITH ACUTE
\DoUTFCombination{00EA}{c3}{^^aa}{\^e} % LATIN SMALL LETTER E WITH CIRCUMFLEX
\DoUTFCombination{00EB}{c3}{^^ab}{\"e} % LATIN SMALL LETTER E WITH DIAERESIS
\DoUTFCombination{00EC}{c3}{^^ac}{\`i} % LATIN SMALL LETTER I WITH GRAVE
\DoUTFCombination{00ED}{c3}{^^ad}{\'i} % LATIN SMALL LETTER I WITH ACUTE
\DoUTFCombination{00EE}{c3}{^^ae}{\^i} % LATIN SMALL LETTER I WITH CIRCUMFLEX
\DoUTFCombination{00EF}{c3}{^^af}{\"i} % LATIN SMALL LETTER I WITH DIAERESIS
\DoUTFCombination{00F0}{c3}{^^b0}{\dh} % LATIN SMALL LETTER ETH
\DoUTFCombination{00F1}{c3}{^^b1}{\~n} % LATIN SMALL LETTER N WITH TILDE
\DoUTFCombination{00F2}{c3}{^^b2}{\`o} % LATIN SMALL LETTER O WITH GRAVE
\DoUTFCombination{00F3}{c3}{^^b3}{\'o} % LATIN SMALL LETTER O WITH ACUTE
\DoUTFCombination{00F4}{c3}{^^b4}{\^o} % LATIN SMALL LETTER O WITH CIRCUMFLEX
\DoUTFCombination{00F5}{c3}{^^b5}{\~o} % LATIN SMALL LETTER O WITH TILDE
\DoUTFCombination{00F6}{c3}{^^b6}{\"o} % LATIN SMALL LETTER O WITH DIAERESIS
\DoUTFCombination{00F7}{c3}{^^b7}{\tcsym{"F6}} % DIVISION SIGN
\DoUTFCombination{00F8}{c3}{^^b8}{\o} % LATIN SMALL LETTER O WITH STROKE
\DoUTFCombination{00F9}{c3}{^^b9}{\`u} % LATIN SMALL LETTER U WITH GRAVE
\DoUTFCombination{00FA}{c3}{^^ba}{\'u} % LATIN SMALL LETTER U WITH ACUTE
\DoUTFCombination{00FB}{c3}{^^bb}{\^u} % LATIN SMALL LETTER U WITH CIRCUMFLEX
\DoUTFCombination{00FC}{c3}{^^bc}{\"u} % LATIN SMALL LETTER U WITH DIAERESIS
\DoUTFCombination{00FD}{c3}{^^bd}{\'y} % LATIN SMALL LETTER Y WITH ACUTE
\DoUTFCombination{00FE}{c3}{^^be}{\th} % LATIN SMALL LETTER THORN
\DoUTFCombination{00FF}{c3}{^^bf}{\"y} % LATIN SMALL LETTER Y WITH DIAERESIS

%%%% Other characters omitted
%%%%%%%%%%%%%%%%%%%%%%%%%%%%%%%%%%%%% END UTF8PLAINMAC %%%%%%%%%%%%%%%%%%%%%%%%%%%%%%%%%%
\message{nobeamer: finished loading utf8plainmac.}

% Slide sizes
\ifpdf
    \pdfpagewidth=8 true in
    \pdfpageheight=6 true in
\fi
\paperwidth=8 true in
\paperheight=6 true in
\topmargin=.2 true in
\bottommargin=.2 true in
\leftmargin=.4 true in
\rightmargin=.4 true in

% No page numbering
\makeatletter
\let\nobeamer@folio\folio
\def\disablepagenumbering{\def\folio{}}
\def\enablepagenumbering{\let\folio\nobeamer@folio}
\makeatother

\disablepagenumbering

% Misc. page setup
\parindent=0pt

% Fonts
\makeatletter
{\ifundefined{scfam}{\newfam\scfam}\fi}
\def\nobeamer@buildfont#1#2#3{%
    \expandafter\font\csname #1#2\expandafter\endcsname=cm#3}
\def\nobeamer@buildfamily#1#2#3#4{%
    % Text fonts
    \nobeamer@buildfont{#1}{rm}{r#2}%
    \nobeamer@buildfont{#1}{it}{ti#2}%
    \nobeamer@buildfont{#1}{sl}{sl#2}%
    \nobeamer@buildfont{#1}{tt}{tt#2}%
    \nobeamer@buildfont{#1}{bf}{bx#2}%
    \nobeamer@buildfont{#1}{sc}{csc#2}%
    % Math fonts
    \def\nobeamer@makemathfonts##1##2{%
        \nobeamer@buildfont{#1}{tf##1}{##2#2}%
        \nobeamer@buildfont{#1}{sf##1}{##2#3}%
        \nobeamer@buildfont{#1}{ssf##1}{##2#4}}%
    \nobeamer@makemathfonts{rm}{r} % Roman
    \nobeamer@makemathfonts{mi}{mi} % Math italic
    \nobeamer@makemathfonts{sy}{sy} % Math symbols
    \nobeamer@makemathfonts{ex}{ex} % Math extension
    \nobeamer@makemathfonts{it}{ti} % Italic
    \nobeamer@makemathfonts{sl}{sl} % Slanted
    \nobeamer@makemathfonts{bf}{bx} % Bold
    \nobeamer@makemathfonts{tt}{tt} % Typewriter
    \nobeamer@makemathfonts{sc}{csc} % Small Caps
    \def\nobeamer@setfonts##1##2##3{%
        \expandafter\textfont##2=\expandafter\csname ##1tf##3\endcsname%
        \expandafter\scriptfont##2=\expandafter\csname ##1sf##3\endcsname%
        \expandafter\scriptscriptfont##2=\expandafter\csname ##1ssf##3\endcsname}%
    \def\nobeamer@oldstyle{rm}
    \expandafter\def\csname #1\endcsname{%
        \nobeamer@setfonts{#1}{0}{rm}%
        \nobeamer@setfonts{#1}{1}{mi}%
        \nobeamer@setfonts{#1}{2}{sy}%
        \nobeamer@setfonts{#1}{3}{ex}%
        \nobeamer@setfonts{#1}{\noexpand\itfam}{it}%
        \nobeamer@setfonts{#1}{\noexpand\slfam}{sl}%
        \nobeamer@setfonts{#1}{\noexpand\bffam}{bf}%
        \nobeamer@setfonts{#1}{\noexpand\ttfam}{tt}%
        \nobeamer@setfonts{#1}{\noexpand\scfam}{sc}%
        \def\styleswitch{%
            \def\scriptsize{\expandafter\csname #1sf\nobeamer@oldstyle\endcsname}%
            \def\scriptscriptsize{\expandafter\csname #1ssf\nobeamer@oldstyle\endcsname}%
            \expandafter\csname #1\nobeamer@oldstyle\endcsname}%
        \def\rm{%
            \def\nobeamer@oldstyle{rm}%
            \styleswitch}%
        \def\it{%
            \def\nobeamer@oldstyle{it}%
            \styleswitch}%
        \def\sl{%
            \def\nobeamer@oldstyle{sl}%
            \styleswitch}%
        \def\bf{%
            \def\nobeamer@oldstyle{bf}%
            \styleswitch}%
        \def\tt{%
            \def\nobeamer@oldstyle{tt}%
            \styleswitch}%
        \def\sc{%
            \def\nobeamer@oldstyle{sc}%
            \styleswitch}%
        \styleswitch%
        \baselineskip=1em}%
    }

\nobeamer@buildfamily%
    {big}{18 scaled 2000}{16 scaled 2000}{15 scaled 2000}
\nobeamer@buildfamily%
    {large}{12 scaled 1600}{10 scaled 1600}{9 scaled 1600}
\nobeamer@buildfamily%
    {regular}{12 scaled 1200}{10 scaled 1200}{9 scaled 1200}
\nobeamer@buildfamily%
    {small}{12}{10}{9}
\nobeamer@buildfamily%
    {smaller}{10}{8}{7}
\nobeamer@buildfamily%
    {tiny}{8}{6}{5}
\makeatother

\regular % is default.

% utility commands
\def\newslide{\vfill\break}
\def\topfill{\topglue 0pt plus 1fill}
\def\fold{\vskip 0pt plus -1fill}
\let\foldbottom\fold % Backwards compatibility

\def\begincenter{%
  \par
  \begingroup
  \leftskip=0pt plus 1fil
  \rightskip=\leftskip
  \parindent=0pt
  \parfillskip=0pt
}
\def\endcenter{%
  \par
  \endgroup
}
\long\def\centering#1{\begincenter#1\endcenter}

\def\mathcal#1{{\fam=2 #1}}
\def\mathrm#1{{\fam=0 #1}}

\message{nobeamer: finished loading. Have a nice day!}
